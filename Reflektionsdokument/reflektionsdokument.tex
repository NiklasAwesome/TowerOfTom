\documentclass{TDP005mall}



\newcommand{\version}{Version 0.1}
\author{Niklas Åsberg, \url{nikas214@student.liu.se}}
\title{Reflektionsdokument}
\date{2023-12-10}
\rhead{Niklas Åsberg}
\usepackage{graphicx}
\graphicspath{{./}}


\begin{document}
\projectpage
\section{Revisionshistorik}
\begin{table}[!h]
\begin{tabularx}{\linewidth}{|l|X|l|}
\hline
Ver. & Revisionsbeskrivning & Datum \\\hline
0.1 & Utkast & 2023-12-10 \\\hline
\end{tabularx}
\end{table}

\tableofcontents
\newpage

\section{Det som varit svårt}
Det var en hel del svårigheter för mig i den här kursen. 
\subsection{C++}
Då jag inte riktigt har programmerat i C++ sen jag läste TDP004 för åtta år sen 
så var det de första stora hindret. Precis innan den här kursen så hade jag en kurs där jag använde 
Java och det tog lite tid att vänja sig vid skillnaderna. \\
\subsection{Läsa flera kurser samtidigt}
Jag har samtidigt som den här kursen läst ett par andra kurser så det var svårt 
att jobba ihop med någon annan då det hade varit i princip omöjligt att få till ett 
schema som skulle fungerat för båda, det gjorde att jag gjorde projektet själv. \\
\subsection{Dokument}
I och med att kursen har många dokument att skriva så har en stor del av tiden lagts på dem, 
jag tycker personligen att skriva den typen av dokument är mycket svårt och att veta var de 
bör innehålla krävdes mycket tid och efterforskande. Dokumentationen gjordes inte lättare av att 
de skulle skrivas i Latex, ett format jag inte ens har tänkt på sen circa 2015. Latex i sig är 
väldigt stort och kompliserat vilket var mycket att lära sig utöver allt annat. \\
\subsection{Intensitet}
Kursen är extremt intensivt, om man som jag har gjort läser flera andra kurser samtidigt så blir 
det lätt att dagens timmar inte riktigt räcker till. \\
Jag hade även svårt att förstå vissa instruktioner för uppgifterna, framför allt instruktionerna 
för dokumenten.


\section{Hur jag hanterade svårigheterna}
\subsection{C++}
Att det var länge sen jag programmerade i C++ hade en ganska simpel lösning, att läsa på. Jag 
gick igenom några tutorials och läste mycket dokumentation, en simpel lösning men ganska tidskrävande. \\
\subsection{Läsa flera kurser samtidigt}
När man läser flera kurser samtidigt är det viktigt att få till en bra planering och att prioritera 
rätt. Här hade jag kunnat göra ett bättre jobb, men jag anpassade mig efter de deadlines som var givna 
och det underlättade planneringen. \\
Det var svårt att på förhand avgöra hur lång tid varje moment skulle ta 
så flera gånger blev det så att dagarna innan deadlinesen så fick jag jobba långt in på natten för 
att hinna med. \\
\subsection{Dokument}
Dokumentationen fick jag lösa genom att skriva om flera gånger. När jag hade skrivit det första utkastet 
så fick jag gå igenom kraven på kursidan och skriva om, efter att ha gjort det några gånger så 
fick jag nöja mig. \\
Svårigheterna med Latex löstes på liknande sätt som C++, läsa dokumentation och 
testa sig fram. Där instruktionerna var svåra att förstå så fick jag fråga labbassarna, men det tog ibland 
tid att få ett svar. Ibland fick jag bara anta vad instruktionerna betydde.

\section{Det jag lärt mig av kursen}
Hur C++ fungerar och används är nog det jag lärt mig absolut mest om. Då jag sedan innan var ganska 
bekant med objektorienterad programmering så var det inte så mycket nytt jag lärde mig inom det. \\
Latex lärde jag mig en hel del om, framför allt själva tex-språket. \\
SFML var helt nytt för mig och verkar vara ett riktigt bra biblotek. Det fungerar väldigt bra ihop 
med objektorienteringen. \\
Jag har lärt mig en del hårda läxor med, mer specifikt kring planering, prioritering och kommunikation. \\
Hade jag läst kursen igen så hade jag nog inte jobbat ensam, jag tror att det hade gått bättre om 
jag hade någon att bolla idéer med. Jag hade även försökt att prata mera med labbassarna om saker 
som jag kört fast på. 

\end{document}
